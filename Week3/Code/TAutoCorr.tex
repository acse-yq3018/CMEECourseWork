\documentclass[12pt]{article}
\title{TAutoCorr}
\author{Yuxin Qin}
\date{Oct 2018}
\begin{document}
  \maketitle

  \section{Question}
    Are temperatures of one year significantly correlated with the next year (successive years), across years in a given location?

  \section{Materials \& Methods}
    I first draw the scatter plot of temperature in each year to glimpse the trend of the temperature during these year.
    Then I used "pearson method" of cor() function in R to calculate the coefficients and p-value. Pearson correlation coefficient (PCC) is also referred to as Pearson's r, which is a measure of the linear correlation between two variables X and Y. 

  \section{Discussion}
    It is unable to observe the trend of temperature by eye via the scatterpoint. 
    The Pearson's r value calculated via cor1() in R is 0.33. The positive of r indicated temperature of successive year has the positive relationship with the temperature of this year. 
    The P-value is 5e-04, which is smaller than 0.05, indicating temperatures of one year significantly correlated with the successive years, across years in a given location.
    The positive pearson r value and the P-value smaller than 0.05 in somehow implicates the increasing temperature in West during these year.

  \section{Results}
    In conclusion, temperatures of one year significantly correlate with the successice years across years in a given location. 
  
\end{document}\grid
